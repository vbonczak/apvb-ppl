\documentclass[svgnames]{article}

\usepackage[utf8]{inputenc}
\usepackage[T1]{fontenc}
\usepackage[french]{babel}
\usepackage[a4paper,margin=1in]{geometry}

\usepackage{vincent} 
\usepackage{listings}

\usepackage{lmodern}

\lstset{
basicstyle=\ttfamily,
keywords = [1]{method, count},
keywords=[2]{let,dist, if, then, else, endif,for,to,do,done},
keywords=[3]{sample, infer,observe, print, factor},
otherkeywords={% Operators
    |, ->, !,^
  },
sensitive=true,  
morecomment=[s]{(*}{*)}, 
keepspaces=true,
morestring=[b]",
keywordstyle=[2]\color{blue},
keywordstyle=[3]\color{DarkRed},
keywordstyle=\color{violet},
stringstyle=\color{DarkGreen},
tabsize=4,
showspaces=false,
breaklines=true
}
 

\begin{document}

\title{Un langage probabiliste basé sur Caml}

\author{Adam Phillips, Vincent Bonczak}

\date{MPRI 2021-2022}

\maketitle

\begin{abstract}
Ce document est le rapport sur le projet du cours de \emph{Langages de Programmation Probabilistes} (cours 2.40 du MPRI). 
Le projet permet d'appliquer les algorithmes d'inférence et les méthodes d'analyse des langages probabilistes rencontrés durant le cours. 
Notre développement s'est concentré sur l'implémentation d'un petit langage inspiré de Caml pour écrire des programmes, compilés ensuite vers
l'OCaml, disposant d'un environnement  d'exécution comportant plusieurs méthodes d'inférence.
\end{abstract}

\section{Travail réalisé}

Nous avons implémenté le projet sous la forme d'un compilateur d'un langage simple vers l'OCaml, compilé avec une bibliothèque de fonctions d'inférence.
Ce langage s'inspire de Caml pour les appels et les instructions de contrôle, en le simplifiant. Par exemple le programme suivant peut être utilisé pour compiler l'exemple \verb|funny_bernoulli| vu en cours :
\begin{lstlisting}
method Rejection

let funny_bernoulli  () =  
 let a = sample (bernoulli 0.5);
 let b = sample (bernoulli 0.5);  
 let c = sample (bernoulli 0.5); 
assume (a=1||b=1)
a+b+c
;

let _ = 
print ".-- Funny Bernoulli, Basic Rejection Sampling --."
dist d = infer funny_bernoulli;
print d
;
\end{lstlisting}

 
Plusieurs constructions probabilistes comme \verb|assume| ou \verb|infer| sont supportées, ainsi qu'une fonction d'affichage rapide de résultats \verb|print|.
On peut modifier le nombre d'échantillons dans plusieurs méthodes d'inférence à l'aide de la commande \lstinline|count << <n>|.
On change la méthode avec \lstinline{method <nom>}, parmi 

Nous avons opté pour un langage à parser pour pouvoir faire du typage et de l'analyse statique (cependant par manque de temps nous n'avons implémenté que le typage simple, l'analyse statique s'étant révélée chronophage à développer).
Le typage vérifie que l'on échantillonne bien une distribution par exemple (soit une des valeurs existantes dans le langage comme \verb|bernoulli| ou une distribution issue d'une inférence, introduite par le mot-clé \lstinline{dist}).



\section{Metropolis-Hastings}

L'inférence \emph{MultiSite} et \emph{SingleSite} est appelée   avec respectivement \lstinline{method MetroMulti}  et \lstinline{method MetroSingle}.


Pendant le parcours du programme en entrée, on ajoute des instructions responsables du traçage des variables aléatoires échantillonnées.
Afin de gérer les dépendances, les appels à \lstinline{sample} seront accompagnés du nom de la variable, et de la position de celle-ci dans le programme en cas de variables de même nom. La réflexion ainsi introduite sur notre langage permettra au \emph{runtime} Caml de savoir quelles variables rééchantillonner et suivre, dans le cas \emph{SingleSite}.

Par exemple :


Ces infos seront données à la fonction sample à chaque appel
ainsi sample peut choisir de rééchantilloner les vars de son choix, selon si c'est first ou resample

\section{La simulation}

 

\section{Conclusion}
 

Nous avons tenu à faire un langage différent de Caml (et plus simple, à l'image de WebPPL) mais cela nous a pris plus de temps que prévu.

\end{document}